\documentclass{article}

\usepackage[utf8]{inputenc}
\usepackage{graphicx}
\usepackage{geometry}
\usepackage{fancyhdr}
\usepackage{xcolor}
\usepackage{anyfontsize}
\usepackage{hyperref}
\usepackage{url}
\usepackage{enumitem}
\usepackage{booktabs}
\usepackage{makecell}
\usepackage{array}
\usepackage{longtable}
\usepackage[sfdefault,light]{roboto} 
\usepackage[T1]{fontenc}

\geometry{
    top=0.8in,
    bottom=0.6in,
    left=0.7in,
    right=0.7in
}

\definecolor{linkblue}{HTML}{0563C1}
\hypersetup{
    colorlinks=true,        % false = boxed links
    linkcolor=linkblue,     % internal links
    urlcolor=linkblue,      % external URLs
    citecolor=linkblue
}
\renewcommand\UrlFont{\color{black}\sffamily}

\fancypagestyle{firstpage}{
    \fancyhf{}
    \fancyhead[C]{
        \noindent
        \begin{minipage}[c]{0.15\textwidth}
            \centering\includegraphics[width=\textwidth]{ {{ instaar_logo_path}} }
        \end{minipage}
        \hfill
        \begin{minipage}[c]{0.2\textwidth}
            \includegraphics[width=1.3\textwidth]{ {{ bureau_reclamation_logo_path }} }
        \end{minipage}
        \hfill
        \begin{minipage}[c]{0.3\textwidth}
            \centering\includegraphics[width=1.1\textwidth]{ {{ cu_boulder_logo_path }} }
        \end{minipage}
    }
    \renewcommand{\headrulewidth}{0pt} 
}

\begin{document}

\thispagestyle{firstpage}
\vspace*{3em}

\fontsize{14}{18}
\begin{center}
    \textbf{\textcolor[HTML]{1F497D}{
        Real-Time Spatial Estimates of Snow-Water Equivalent (SWE)\\
    }}
    \textbf{
        Western United States Region\\
        {{ date_string }}\\
        \vspace{1em}
    }
\end{center}

\fontsize{10}{14}
\noindent\textbf{Team:} Noah Molotch$^{1,2}$, Karl Rittger$^1$, Leanne Lestak$^1$, Emma Tyrrell$^1$, and Kehan Yang$^1$\\
$^1$ Institute of Arctic and Alpine Research, University of Colorado Boulder\\
$^2$ Jet Propulsion Laboratory, California Institute of Technology\\
\textit{Report generation funded by: U.S. Bureau of Reclamation}\\
\textit{Contact: Leanne.Lestak@colorado.edu}\\

\fontsize{12}{16}
\noindent\textbf{\textit{Introduction}}\\
\fontsize{10}{14}
Figure 1 below displays estimated SWE amounts across the Western United States. Detailed SWE maps (in JPG format) and summaries of SWE (in Excel format) by individual basin and elevation band accompany the report and are publicly available \href{https://github.com/CU-Mountain-Hydrology/WestWide}{here}. Please note that the basin-wide percent of long-term average from the spatial SWE estimates is not directly comparable with the SNOTEL basin-wide percent of average. A better comparison might be made with the percent of average in the elevation banded tables (linked below) that contain SNOTEL sites.\\
\section*{}\label{fig:1a}\vspace{-4em}
\section*{}\label{fig:1b}\vspace{-4em}
\noindent
\begin{minipage}[t]{0.5\textwidth}
    \centering \includegraphics[width=\textwidth]{ {{ fig1a_path }} }
\end{minipage}
\noindent
\begin{minipage}[t]{0.5\textwidth}
    \centering\includegraphics[width=\textwidth]{ {{ fig1b_path }} }
\end{minipage}
\textbf{\textit{Figure 1. Estimated SWE and \% of Average SWE across the Western U.S.}} SWE amounts across the entire Western region of the United States (left) and percent of long-term average (2001-2021) by five regions (right). Region boundaries are delineated based on Snowpack regimes of the Western United States (Trujillo and Molotch, 2014) and the Commission for Environmental Cooperation (CEC) Ecological Regions of North America, Level III [Commission for Environmental Cooperation, 2009, available at \href{http://www.cec.org/north-american-environmental-atlas/terrestrial-ecoregions-level-iii/}{http://www.cec.org/north-american-environmental-atlas/terrestrial-ecoregions-level-iii/}].\\
\noindent
\begin{minipage}[t]{0.5\textwidth}
    \centering \includegraphics[width=\textwidth]{ {{ fig2a_path }} }
\end{minipage}
\noindent
\begin{minipage}[t]{0.5\textwidth}
    \centering\includegraphics[width=\textwidth]{ {{ fig2b_path }} }
\end{minipage}
\textbf{\textit{Figure 2. Estimated \% of Average SWE across the Western U.S.}} Percent of long-term average (2001-2021) from the spatial SWE calculated for each pixel (left) and by HUC-6 basin (right); integer within each watershed represents the percent of average SWE for the report date. Shaded areas (right) correspond to the elevation bands used in the tables below.\\

\noindent\textbf{\textit{For detailed maps and tabular summaries of SWE and snowpack water storage volumes for specific regions and watersheds, click on the links below:}}\\
\hyperref[sec:pnw]{\textit{\underline{Pacific Northwest}}}\\
\hyperref[sec:nocn]{\textit{\underline{North Continental}}}\\
\hyperref[sec:socn]{\textit{\underline{South Continental}}}\\
\hyperref[sec:inmt]{\textit{\underline{Intermountain}}}\\
\hyperref[sec:snm]{\textit{\underline{Sierra Nevada}}}\\
\hyperref[sec:elev_band_swe]{\textit{\underline{Elevation Banded SWE Tables}}}\\

\noindent\textbf{\textit{About this report}}\\
This is an experimental research product that provides near-real-time estimates of snow-water equivalent (SWE) at a spatial resolution of 500 meters for the Western region of the United States from mid-winter through the melt season. The report is typically released within a week of the date of data acquisition at the top of the report. A similar report covering the Sierra Nevada has been distributed to water managers in California since 2012.\\

\noindent
The spatial SWE data fusion (SWE-fusion) analysis method for the Western U.S. uses the following data as inputs:
\begin{itemize}[label=-]
    \item In-situ SWE from all operational NRCS and CDEC snow pillow sites, and the CoCoRaHS network when appropriate
    \item Fractional snow-covered area (fSCA) data from recent cloud-free satellite images
    \item Physiographic information (elevation, latitude, upwind mountain barriers, slope, etc.)
    \item Historical daily SWE patterns (1985-2021) retrospectively generated using historical fSCA data and an energy-balance model that back-calculates SWE given the fSCA time-series and meltout date for each pixel
    \item Satellite-observed daily mean fractional snow-covered area (DMFSCA)
\end{itemize}

\noindent For more details see the \hyperref[sec:methods]{\textcolor{black}{\textit{Methods}}} section below. Please be sure to read the \hyperref[sec:data_issues]{\textcolor{black}{\textit{Data Issues / Caveats}}} section for a discussion of persistent challenges or flagged uncertainties of the SWE-fusion product.\\

\noindent\textbf{\textit{Data availability for reporting}}\\
Snow pillows located throughout the Western U.S. region are input as the dependent variable in the SWE-fusion system. 799 Natural Resources Conservation Service (NRCS) Snow Telemetry (SNOTEL) sites and 131 California Department of Water Resources (CA-DWR) California Data Exchange Center (CDEC) are potentially available for each model run. In addition, the Community Collaborative Rain, Hail and Snow (CoCoRaHS, https://www.cocorahs.org/) network provides over 500 snow measurements across the modeling domain.\\

\noindent\textbf{\textit{Maps and Tables by Region}}\\
Maps and tables for each of the five western regions (\hyperref[fig:1b]{\textcolor{black}{Figure 1b}}) are shown below. Note that the basin-wide averages may reflect variable conditions across the elevation bands; see banded-elevation tables (linked below). Basin-wide percent of average is calculated across all model pixels inside a given basin and base elevation. Basin base elevations vary anywhere between 2,000’ to 7,000’. Base elevations are dependent on long-term snow coverage. For example, a base elevation in the north could be lower as compared to a base elevation in the south.\\


\newpage
\section*{}\label{sec:pnw}
\fontsize{12}{16}
\noindent\textbf{\textit{Pacific Northwest}}
\fontsize{10}{14}
\begin{center}
\noindent 
\includegraphics[width=\textwidth]{ {{ fig3_path }} }
\end{center}
\noindent\textbf{\textit{Figure 3. Estimated SWE and \% of Average SWE across the Pacific Northwest Region.}} SWE amounts (upper left), percent of long-term average (2001-2021) SWE calculated for each pixel (upper right), basin-wide percent of long-term average (lower left) shaded areas correspond to the elevation bands used in the banded-elevation tables, and basin identification numbers that correspond to Table 1 below (lower right). The North Puget Sound and Upper Columbia basin portions that are inside Canada do not contain SWE-fusion model data due to lack of data availability needed to run the model in Canada.\\

\newpage
\noindent\textbf{\textit{Table 1. SWE by watershed.}} Shown are percent of average SWE to date for the current date (2001-21 as derived from the regression model), mean SWE for the current report, current percent of snow-covered area, current SWE volume (acre-feet), the area (mi$^2$) inside each basin that contains data pixels (not including cloud-covered pixels, lakes or other satellite no data pixels), first of the month snow surveys, and current snow pillow sensors (the number of stations are in parentheses), for those areas collected, summarized for each basin. \hyperref[sec:elev_band_swe]{\underline{\textit{SWE tables by banded elevation are here.}}}
\input{ {{ table1_path }} }

\newpage
\section*{}\label{sec:nocn}
\fontsize{12}{16}
\noindent\textbf{\textit{North Continental}}
\fontsize{10}{14}
\begin{center}
\noindent 
\includegraphics[width=\textwidth]{ {{ fig4_path }} }
\end{center}

\noindent\textbf{\textit{Figure 4. Estimated SWE and \% of Average SWE across the North Continental Region.}} SWE amounts (upper left), percent of long-term average (2001-2021) SWE calculated for each pixel (upper right), basin-wide percent of long-term average (lower left) shaded areas correspond to the elevation bands used in the banded-elevation tables, and basin identification numbers that correspond to Table 2 below (lower right).\\

\newpage
\noindent\textbf{\textit{Table 2. SWE by watershed.}} Shown are percent of average SWE to date for the current date (2001-21 as derived from the regression model), mean SWE for the current report, current percent of snow-covered area, current SWE volume (acre-feet), the area (mi$^2$) inside each basin that contains data pixels (not including cloud-covered pixels, lakes or other satellite no data pixels), first of the month snow surveys, and current snow pillow sensors (the number of stations are in parentheses), for those areas collected, summarized for each basin. \hyperref[sec:elev_band_swe]{\underline{\textit{SWE tables by banded elevation are here.}}}
\input{ {{ table2_path }} }

\newpage
\section*{}\label{sec:socn}
\fontsize{12}{16}
\noindent\textbf{\textit{South Continental}}
\fontsize{10}{14}
\begin{center}
\noindent 
\includegraphics[width=\textwidth]{ {{ fig5_path }} }
\end{center}
\noindent\textbf{\textit{Figure 5. Estimated SWE and \% of Average SWE across the South Continental Region.}} SWE amounts (upper left), percent of long-term average (2001-2021) SWE calculated for each pixel (upper right), basin-wide percent of long-term average (lower left) shaded areas correspond to the elevation bands used in the banded-elevation tables, and basin identification numbers that correspond to Table 3 below (lower right).\\

\newpage
\noindent\textbf{\textit{Table 3. SWE by watershed.}} Shown are percent of average SWE to date for the current date (2001-21 as derived from the regression model), mean SWE for the current report, current percent of snow-covered area, current SWE volume (acre-feet), the area (mi$^2$) inside each basin that contains data pixels (not including cloud-covered pixels, lakes or other satellite no data pixels), first of the month snow surveys, and current snow pillow sensors (the number of stations are in parentheses), for those areas collected, summarized for each basin. \hyperref[sec:elev_band_swe]{\underline{\textit{SWE tables by banded elevation are here.}}}
\input{ {{ table3_path }} }

\newpage
\section*{}\label{sec:inmt}
\fontsize{12}{16}
\noindent\textbf{\textit{Intermountain}}
\fontsize{10}{14}
\begin{center}
\noindent 
\includegraphics[width=\textwidth]{ {{ fig6_path }} }
\end{center}
\noindent\textbf{\textit{Figure 6. Estimated SWE and \% of Average SWE across the South Continental Region.}} SWE amounts (upper left), percent of long-term average (2001-2021) SWE calculated for each pixel (upper right), basin-wide percent of long-term average (lower left) shaded areas correspond to the elevation bands used in the banded-elevation tables, and basin identification numbers that correspond to Table 4 below (lower right).\\

\newpage
\noindent\textbf{\textit{Table 4. SWE by watershed.}} Shown are percent of average SWE to date for the current date (2001-21 as derived from the regression model), mean SWE for the current report, current percent of snow-covered area, current SWE volume (acre-feet), the area (mi$^2$) inside each basin that contains data pixels (not including cloud-covered pixels, lakes or other satellite no data pixels), first of the month snow surveys, and current snow pillow sensors (the number of stations are in parentheses), for those areas collected, summarized for each basin. \hyperref[sec:elev_band_swe]{\underline{\textit{SWE tables by banded elevation are here.}}}
\input{ {{ table4_path }} }


\newpage
\section*{}\label{sec:snm}
\fontsize{12}{16}
\noindent\textbf{\textit{Sierra Nevada}}
\fontsize{10}{14}

\noindent There is a separate SWE report which also includes maps and tables that has a stronger focus on the Sierra Nevada, it is available \href{https://github.com/CU-Mountain-Hydrology/SierraNevada}{\underline{here.}} The Sierra report incorporates additional vetting and can include bias-corrections with Airborne Snow Observatory data. Below is one of the maps from the current report.\\

\textcolor{red}{\textbf{TODO: Sierra Nevada Preview Map}}\\

\noindent\textbf{\textit{Figure 7. Estimated SWE and \% of Average SWE across the Sierra Nevada.}} SWE amounts (left), and percent of average (2001-2021) SWE for the Sierra Nevada, calculated for each pixel (middle) and basin-wide (right). Basin-wide percent of average is calculated across all model pixels >5000’ elevation.\\

\noindent\textbf{\textit{Table 5. SWE by watershed.}} Shown are percent of average SWE to date for the current date (2001-21 as derived from the regression model), mean SWE for the current report, current percent of snow-covered area, current SWE volume (acre-feet), the area (mi$^2$) inside each basin that contains data pixels (not including cloud-covered pixels, lakes or other satellite no data pixels), first of the month snow surveys, and current snow pillow sensors (the number of stations are in parentheses), for those areas collected, summarized for each basin. \hyperref[sec:elev_band_swe]{\underline{\textit{SWE tables by banded elevation are here.}}}

\textcolor{red}{\textbf{TODO: SNM SWE table here}}\\

\newpage
\section*{}\label{sec:elev_band_swe}
\fontsize{12}{16}
\noindent\textbf{\textit{Elevation Banded SWE Tables}}\\
\fontsize{10}{14}
\noindent Due to the length of the banded elevation tables (tables 6-10), that data is being hosted on our GitHub repository. Direct links to all of the tables are below. Access to the GitHub repository for the tables in both HTML and CSV formats is \href{https://github.com/CU-Mountain-Hydrology/WestWide}{\underline{here.}}
\begin{itemize}
    \item \href{https://cu-mountain-hydrology.github.io/WestWide_HTML/PNW_20250331_table06.html}{\underline{Pacific Northwest (Table 6)}}
    \item \href{https://cu-mountain-hydrology.github.io/WestWide_HTML/NOCN_20250331_table07.html}{\underline{North Continental (Table 7)}}
    \item \href{https://cu-mountain-hydrology.github.io/WestWide_HTML/SOCN_20250331_table08.html}{\underline{South Continental (Table 8)}}
    \item \href{https://cu-mountain-hydrology.github.io/WestWide_HTML/INMT_20250331_table09a.html}{\underline{Intermountain, part 1 (Table 9a)}}\\
    \href{https://cu-mountain-hydrology.github.io/WestWide_HTML/INMT_20250331_table09b.html}{\underline{Intermountain, part 2 (Table 9b)}}
    \item \href{https://cu-mountain-hydrology.github.io/WestWide_HTML/SNM_20250331_table10.html}{\underline{Sierra Nevada (Table 10)}}\\
\end{itemize}

\fontsize{12}{16}
\noindent\textbf{\textit{The value of spatially explicit estimates of SWE}}\\
\fontsize{10}{14}
\noindent Snowmelt makes up the large majority (\textasciitilde60-85\%) of the annual streamflow in the Western U.S. The spatial distribution of SWE across the landscape is complex. While broad aspects of this spatial pattern (e.g., more SWE at higher elevations and on north-facing exposures) are fairly consistent, the details vary a lot from year to year, influencing the magnitude and timing of snowmelt-driven runoff.\\

\noindent SWE is operationally monitored at hundreds of NRCS SNOTEL and California DWR CDEC snow pillow sites spread across the Western U.S., providing a critical first-order snapshot of conditions, and the basis for runoff forecasts from the CA DWR, NRCS and NOAA. However, conditions at snow pillow sites (e.g., percent of normal SWE) may not be representative of conditions in the large areas between these point measurements, and at elevations above and below the range of the pillow sites. The spatial SWE-fusion creates a detailed picture of the spatial pattern of SWE using snow pillows, satellite, and other data, extending beyond the snow pillow sites to unmonitored areas.\\

\fontsize{12}{16}
\noindent\textbf{\textit{Interpreting the spatial SWE estimates in the context of snow pillow sites}}\\
\fontsize{10}{14}
\noindent The spatial SWE-fusion product estimates SWE for every pixel where the fractional snow-covered area (fSCA) satellite product identifies snow-cover. Comparatively, snow pillow samples on average 8-20 points per basin within a narrower elevation range. Thus, the basin-wide percent of long-term average from the spatial SWE-fusion estimates is not directly comparable with the snow pillow basin-wide percent of average. A better comparison might be made with the \% average in the elevation bands (\href{https://github.com/CU-Mountain-Hydrology/WestWide}{\underline{elevation-banded tables 6-10}}) that contain snow pillow sites.\\

\fontsize{12}{16}
\noindent\textbf{\textit{Location of Reports, Excel Format Tables, and JPG Maps}}\\
\fontsize{10}{14}
\noindent \href{https://github.com/CU-Mountain-Hydrology/WestWide}{\underline{https://github.com/CU-Mountain-Hydrology/WestWide}}

\section*{}\label{sec:methods}\vspace{-3em}
\fontsize{12}{16}
\noindent\textbf{\textit{Methods}}\\
\fontsize{10}{14}
\noindent The spatial SWE-fusion estimation method is described in Yang, et. al. (2022) and Schneider and Molotch (2016). The method uses a General Linear Model in which the dependent variable is derived from the operationally measured in situ SWE from all online NRCS SNOTEL and CDEC snow pillow sites in the domain and when applicable the CoCoRaHS SWE values. The snow pillow SWE observations are scaled by the satellite-based fractional snow-covered area (fSCA) across the 500-meter pixel containing that snow pillow site before being used in the linear regression model. The fSCA is a near-real-time cloud-free daily satellite image from the Snow Today fSCA image (Rittger, et. al. 2019, \href{https://nsidc.org/snow-today}{\underline{https://nsidc.org/snow-today}}) which uses the SPIReS algorithm (Bair, et al. 2021).\\

\noindent The following independent variables (predictors) enter the linear regression model:
\begin{itemize}[label=-]
    \item Physiographic variables that affect snow accumulation, melt, and redistribution, including elevation, latitude, upwind mountain barriers, slope, and others. See Table 1 in Yang, et. al., (2022) for the full set of these variables.
    \item The historical daily SWE pattern (1985-2021) retrospectively generated using historical Landsat data, and an energy-balance model that back-calculates SWE given the fractional Snow-Covered Area (fSCA) time series and meltout date for each pixel. See Fang, et. al., (2022) for details. (For computational efficiency, only one image during the 1985-2021 period that best matches the real-time snow pillow-observed pattern is selected as an independent variable.)
    \item Satellite-observed daily mean fractional snow-covered area (DMFSCA) derived from Rittger, et. al., (2019) data.
\end{itemize}

\noindent The real-time regression model for this date has been validated by cross-validation, whereby 10\% of the snow pillow data are randomly removed and the model prediction is compared to the measured value at the removed snow pillow stations. This is repeated 30 times to obtain an average R-squared value, which denotes how closely the model fits the snow pillow data. During development of this regression method, the model was also validated against independent historical SWE data from Airborne Snow Observatory lidar data and from snow surveys at 10 locations in Colorado.\\


\section*{}\label{sec:data_issues}\vspace{-3em}
\fontsize{12}{16}
\noindent\textbf{\textit{List of All Known Data Issues/Caveats}}
\fontsize{10}{14}
\begin{itemize}
    \item SATELLITE FSCA - Recent snowpack accumulation particularly in the Arizona / NM region may be under-estimated due to issues with satellite-observed fSCA.
    \item GLACIER \& NON-SEASONAL SNOW – SWE values on non-seasonal snow and glaciers need to be excluded before data analysis.
    \item RECENT SNOWFALL – There are occasionally problems with lower-elevation SWE estimates due to recent snowfall events that result in extensive snow-cover extending to valley locations where measurements are not available. This scenario results in an over-estimation of lower- elevation SWE.
    \item LIMITED SNOW PILLOW DATA – When snow at the snow pillow sites melts out, but remains at higher elevations, the model tends to overestimate SWE at the under-monitored upper elevations. This issue typically occurs late in the melt season, resulting in less accurate SWE prediction at higher elevations compared to earlier in the snow season. 
    \item CLOUD COVER – Cloud cover can obscure satellite measurements of snow-cover. While careful checks are made, occasionally the misclassification of clouds as snow or vice versa may result in the mischaracterization of SWE or bare-ground.
    \item LOW LOOK ANGLE – When a satellite does not pass directly over a region but the area is still included within the satellite sensor’s field of view, this is referred to as a low “look angle”. The resulting image has lower effective resolution – this “blurry” MODSCAG data still contains useful information but may lead to overestimation of SWE near the margins of the snow-cover extent.
    \item POOR QUALITY SNOW SENSOR DATA – Although data QA/QC is performed, occasional SNOTEL sensor malfunction may result in localized SWE errors.
    \item ANOMALOUS SNOW PATTERNS – Anomalous snow years or snow distributions may cause SWE error due to the model design to search for similar SWE distributions from previous years. If no close seasonal analogue exists, the model is forced to find the most similar year, which may result in error.
    \item DENSE FOREST COVER – Dense forest cover at lower elevations where snow-cover is discontinuous can cause the satellite to underestimate the snow-cover extent, leading to underestimation of SWE.
    \item PERCENT OF AVERAGE CALCULATIONS - Data utilized to generate this report change to optimize model performance.  To maintain consistency across the historical record, the percent of average values are based on our baseline algorithm and therefore there can be discrepancies between absolute SWE values and corresponding percent of averages.
    \item MODELING METHODS - We work to generate the best SWE estimates for each reporting date. Our methods can change from one report to another. Sometimes data changes between reports is an artifact of method changes.
    \item EARLY SEASON FSCA ERRORS – The gap-filled fSCA requires some cloud-free images to determine fSCA amounts. Early in the season and if it has been particularly cloudy the algorithm hasn’t had time to calculate fSCA amounts in some areas, typically in the Pacific Northwest and northern areas of the domain.
\end{itemize}

\newpage
\fontsize{12}{16}
\noindent\textbf{\textit{References and Additional Sources}}\\
\fontsize{10}{14}

\noindent
\hangindent=1.5em
Bair, E.H., T. Stillinger and J. Dozier (2021). Snow Property Inversion From Remote Sensing (SPIReS): A Generalized Multispectral Unmixing Approach With Examples From MODIS and Landsat 8 OLI. IEEE Transactions on Geoscience and Remote Sensing, 59(9): 7270-7284. DOI: 10.1109/TGRS.2020.3040328.

\noindent
\hangindent=1.5em
Commission for Environmental Cooperation (2009). Ecological regions of North America, Level 3, scale 1:4,000,000, Commission for Environmental Cooperation, Montreal, Quebec, Canada.

\noindent
\hangindent=1.5em
\textit{Hall, D. K. and G. A. Riggs (2021). MODIS/Terra Snow Cover Daily L3 Global 500m SIN Grid, Version 61. Boulder, Colorado USA. NASA National Snow and Ice Data Center Distributed Active Archive Center. doi: \url{https://doi.org/10.5067/MODIS/MOD10A1.061}. Date Accessed May 10, 2022.}

\noindent
\hangindent=1.5em
Fang, Y., Liu, Y. \& Margulis, S.A. A western United States snow reanalysis dataset over the Landsat era from water years 1985 to 2021 (2022). Sci Data 9, 677. https://doi.org/10.1038/s41597-022-01768-7.

\noindent
\hangindent=1.5em
Molotch, N.P. (2009). Reconstructing snow water equivalent in the Rio Grande headwaters using remotely sensed snow cover data and a spatially distributed snowmelt model. \textit{Hydrological Processes}, Vol. 23, doi: 10.1002/hyp.7206, 2009.

\noindent
\hangindent=1.5em
Molotch, N.P., and S.A. Margulis (2008). Estimating the distribution of snow water equivalent using remotely sensed snow cover data and a spatially distributed snowmelt model: a multi-resolution, multi-sensor comparison. \textit{Advances in Water Resources}, 31, 2008.

\noindent
\hangindent=1.5em
Molotch, N.P., and R.C. Bales (2006). Comparison of ground-based and airborne snow-surface albedo parameterizations in an alpine watershed: impact on snowpack mass balance. \textit{Water Resources Research}, VOL. 42, doi:10.1029/2005WR004522.

\noindent
\hangindent=1.5em
Molotch, N.P., and R.C. Bales (2005). Scaling snow observations from the point to the grid-element: implications for observation network design. \textit{Water Resources Research}, VOL. 41, doi: 10.1029/2005WR004229.

\noindent
\hangindent=1.5em
Molotch, N.P., T.H. Painter, R.C. Bales, and J. Dozier (2004). Incorporating remotely sensed snow albedo into a spatially distributed snowmelt model. \textit{Geophysical Research Letters}, VOL. 31, doi:10.1029/2003GL019063, 2004.

\noindent
\hangindent=1.5em
Rittger, K., M. S. Raleigh, J. Dozier, A. F. Hill, J. A. Lutz, and T. H. Painter (2019). Canopy Adjustment and Improved Cloud Detection for Remotely Sensed Snow Cover Mapping. Water Resources Research 24 August 2019. doi:10.1029/2019WR024914.

\noindent
\hangindent=1.5em
Schneider D. and N.P. Molotch (2016). Real-time estimation of snow water equivalent in the Upper Colorado River Basin using MODIS-based SWE reconstructions and SNOTEL data. \textit{Water Resources Research}, 52(10): 7892-7910. DOI: 10.1002/2016WR019067.

\noindent
\hangindent=1.5em
Trujillo, E., and N. P. Molotch (2014). Snowpack regimes of the Western United States, Water Resour. Res., 50, 5611–5623, doi:10.1002/ 2013WR014753.

\noindent
\hangindent=1.5em
Yang, K., K. N. Musselman, K. Rittger, S. A. Margulis, T. H. Painter and N. P. Molotch (2022). Combining ground-based and remotely sensed snow data in a linear regression model for real-time estimation of snow water equivalent. \textit{Advances in Water Resources}, 160, 2022, 104075. DOI: 10.1016/j.advwatres.2021.104075.

\end{document}
